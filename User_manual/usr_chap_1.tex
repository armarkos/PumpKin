%=============================================================================
%=============================================================================

\chapter{Introduction}
\label{chap_1}
\pump{} is a user-friendly software package to find all principal pathways, i.e. the dominant reaction sequences, in chemical reaction systems. The goal is to analyze the production and/or destruction mechanisms of a certain species of interest, as well as to reduce a complex plasma chemistry models.

\pump{} was developed by Aram H. Markosyan at CWI (Centrum Wis\-kunde \& Informatica), Amsterdam under the STW project 10751 ``Transient plasmas for air purification'' and at IAA-CSIC (Instituto de Astrofísica de Andalucía - CSIC), Granada during short visits of A.H. Markosyan to Dr. F.J. Gordillo-V\'{a}zquez and Dr. A. Luque under the ESF (European Science Foundation) grants 5697, 5698, 5297 within the TEA-IS (Thunderstorm effects on the atmosphere-ionosphere system) activities. A. Luque contributed to the checking and validation of the code.

\pump{} is free software; you can redistribute it and/or modify it under the terms of the GNU General Public License (as published by the Free Software Foundation) version 2. \pump{} can be downloaded from the following address: \url{www.pumpkin-tool.org}.

You should have received a copy of the GNU General Public License along with this program; if not, contact Aram H. Markosyan at \url{armarkos@umich.edu} or write to the Free Software Foundation, Inc., 59 Temple Place, Suite 330, Boston, MA 02111-1307 USA.

%=============================================================================

\section{System Requirements}
\label{sec_1_1}

\pump{} is written in the C++ programming language. It has been tested on Mac OS X, Linux OS and Microsoft Windows. A C++ compiler is required. We have tested \pump{} with the following compilers: GCC and LLVM.  We recommend Windows users to use Cygwin (\url{www.cygwin.com}), which implements a GNU toolchain in the Windows architecture. In general here are the general requirements:
\begin{itemize}
\item {To build \pump{}, the GNU version of make (GNUmake) must be installed. The \pump{} makefile requires GNU make version 3.77 or later. GNU software can be downloaded from many places, including \url{www.gnu.org/software/make/}.}
\item {A C++ compiler is required. \pump{} makes heavy use of the ISO/IEC 14882 C++ Standard. Some compilers are not fully compliant with this specification, although most are. \pump{} has been compiled and tested with
\begin{itemize}
\item {GNU g++ 3.32 or higher.}
\item {LLVM 3.2 or higher}
\end{itemize}}
\item {GLPK (GNU linear programming kit) must be installed~\cite{Glpk}. This allows \pump{} to solve large-scale linear programming (LP) problems. GLPK can be downloaded from \url{www.gnu.org/software/glpk/}. We have tested \pump{} with GLPK version higher than 4.32.}
\end{itemize}
The recommended system requirements depend on the choice of the input parameters and the problem size. As a reference, in a MacBook Pro 15-inch (Mid 2010) with a CPU Intel Core i5 at 2.4 GHz, 4 GB (1067 MHz DDR3) of RAM memory and the operating system Mac OS X 10.9, \pump{} runs the examples from the section~\ref{chap_4} in about 30 seconds. When the input files are large, \pump{} will require more time to load them into the computer memory.





























