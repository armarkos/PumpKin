%=============================================================================
%=============================================================================

\chapter{Input and Output}
\label{chap_3}

%=============================================================================

\section{Input}
\label{sec_3_1}

To determine the chemical pathways, \pump{} requires from the user the stoichiometric 
matrix and kinetic data for the full chemical reaction system, namely:
\begin{itemize}
\item {chemical reactions $R_j$, $j = 1, \ldots, n_R$, involving between 
species $S_i$, $i = 1,\ldots, n_S$,
where $n_R$ and $n_S$ are the number of chemical reactions and species, respectively,
}
\item {stoichiometric coefficients $s_{ij}$, which represent the number of 
molecules of species $S_i$ produced (or negative number of molecules of 
$S_i$ consumed) by reaction $R_j$,}
\item{a time evolution of concentrations $c_i(t_l)$ and reactions rate $r_j(t_l)$, 
where $l = 1, \ldots, n_T$ and $t_0 = t_1\leq\ldots \leq t_{n_T}=t_{end}$,}
\end{itemize}
The code is independent of the units chosen by the user.  Conventionally, $c_i(t_l)$ is specified in units of [molec. cm$^{-3}$] and $r_j$ in units of [molec. cm$^{-3}$ s$^{-1}$].

\pump{} expects that user stores data in the following files:
\begin{description}
  \item{\verb|qt_species_list.txt|:} Contains the names of species included in the model. 
  \item{\file{qt_reactions_list.txt}} Contains human-readable reaction signatures.  \item{\file{qt_conditions.txt}} Contains the time steps $t_l$ (see \ref{sec_6_1}) resulting from the simulation.
  \item{\file{qt_matrix.txt}} Contains the stoichiometric matrix of the chemical model.
  \item{\file{qt_densities.txt}} Contains the time-dependent densities of each species at times $t_l$.  
  \item{\file{qt_rates.txt}} Contains the time-dependent rates of each reaction at times $t_l$.  
  \item{\file{input.txt}} The user should also provide an input file similar to the following table
\begin{align*}
\texttt{interest} &\texttt{ =  1} \\
\texttt{t\_init} & \texttt{ = 0.0}\\ 
\texttt{t\_end} & \texttt{ = 1.0e-3}\\ 
\texttt{max\_bp} & \texttt{ = 0}\\ \label{eq:aram} \tag{*} 
\texttt{tau\_lifetime} & \texttt{ = 0.9e-5}\\ 
\texttt{max\_path} & \texttt{ = 1500}\\
\texttt{f\_min} & \texttt{ = 0}\\
\texttt{global\_kin} & \texttt{ = 1,}
\end{align*}
where:  
\begin{itemize}
\item{\code{interest} - an index of the species of interest $S_{\verb"interest"}$, if the user is interested in the production and/or consumption of $S_{\verb"interest"}$.  Otherwise the user should specify \code{interest} as a non-positive number,}
\item{[\code{t\_init}, \code{t\_init}] - a time interval [\code{t\_init}, \code{t\_init}]$\subseteq[0,T]$ where 
\pump{} will perform the analysis,}
\item{\code{max\_bp} - if positive, the maximum number of branching points considered, otherwise it is disregarded,}
\item{\code{tau\_lifetime} - if positive, a lifetime threshold with units of [s], otherwise it is disregarded,}
\item{\code{max\_path} - if positive, the maximum number of pathways considered per branching point treatment, i.e. only the first \code{max\_path} pathways with higher rate will be considered, otherwise it is disregarded,}
\item{\code{f\_min} - if positive, pathway rate threshold in units of [molec. cm$^{-3}$$s^{-1}$], i.e. pathways with a rate smaller than \code{f\_min} will be deleted, otherwise it is disregarded.}
\item{\code{global\_kin} - boolean parameter. If \code{1} (or \code{true})} \pump{} will interpret the input files as from Global\_Kin, otherwise from ZDPlaskin.
\end{itemize}
\end{description}

The order of parameters in the input file should be exactly like in the table (\ref{eq:aram}). The names of parameters are not important. On the other hand, the names of input files are very important. In order to keep compatibility with VMS/VAX systems, the input files can be all in capitals, except the \file{input.txt}. Currently, \pump{} 
is fully compatible with the output formats of ZDPlasKin~\cite{Zdplaskin,Flitti2009/EPJAP,Qtplaskin} and Global\_Kin.  The \pump{} package 
is distributed examples of input files. 

%=============================================================================

\section{Output}
\label{sec_3_2}

Depending on whether the user has specified the \code{interest} parameter as a positive integer (the index of the species of interest) or as a non-positive number (the user does not have any species of interest), one of the following results will be printed:

\code{interest} $> 0$: \pump{} will output all the pathways (and their rates) producing or consuming the species of interest $S_{\verb"interest"}$, as well as the relative production or consumption compared with the initial concentration of $S_{\verb"interest"}$. The output will also contain information such as how much $S_{\verb"interest"}$ has been produced or consumed by the pathways that are deleted by \pump{} using parameters \code{f\_min} or \code{max\_path}. 

\code{interest} $\leq 0$: \pump{} will output all the pathways (and their rates) sorted by rate. In some cases, this number can be very large, so we decided to limit it by 100, which of course can be easily changed inside the \pump{} source code. The output will also contain information such as the amount of a certain species that has been produced or consumed by the pathways that are deleted by \pump{} according to the parameters \code{f\_min} or \code{max\_path}. 













